\chapter{总结与展望}

	\section{问题讨论}

		\subsection{性能问题}
		
			该研究增加了基于静态策略的安全功能,并在添加安全功能处理时间和安全功能后比较文件大小。基于权限的研究不提供方法级安全功能,因此很难直接比较性能。

			基于静态策略的安全功能研究并未在整个方法单元中添加安全功能,而是仅根据预设策略将安全功能添加到需要安全性的方法位置。相反,建议的框架为AndroidManifest.xml中声明的所有活动类的方法单元增加了安全性。然而,所提出的框架的安全补丁的处理时间比先前的研究(将安全功能添加到平均0.15秒的两种方法)从最小1秒到最大2.6秒更长。但是,考虑到方法单元的所有添加,开销很小。

			在文件大小的情况下,原始应用程序中修补应用程序的大小更改比基于静态策略的研究高约1.5%。但是,考虑到所有方法都添加了所有安全方法,应用程序的最小大小已经增加,并且由于使用Java反射技术添加了简化的安全功能,仅增加了应用程序大小的2.05%。此外,提供管理界面以检查策略管理和应用程序进度流,从而提高用户便利性。
		
		\subsection{实验限制}
		
			在实验应用程序中,排除了具有自己的安全功能(修改预防)的20个应用程序。由于冗余,未考虑向安全内置应用程序添加安全功能。修改防止功能是不正确地反编译app的功能和在执行app之后检查签名密钥值的功能。在后一种情况下,可以在通过绕过签名密钥值验证逻辑禁用伪造防止功能之后添加安全功能。但是,没有执行绕过这些应用程序的防伪功能,因为它可能看起来是恶意的。

		\subsection{易受攻击的代码补丁}
	
			建议的框架显示了如何为不安全的应用程序添加安全功能。在以前的研究中,已经有研究在字节码级别找到并修补误用的加密API\cite{Ma_cdrep_2016}。虽然本文未涉及,但也可以通过检测在检查应用程序的进度流程或者代码中需要安全更新的过程中需要安全更新的情况来更新代码。例如,它是用SSL(安全通信)代码更新在一般通信中实现的代码的功能。

		\subsection{法律问题}
	
			确定该应用是否违反了Google Play的政策,禁止在没有Android源代码的情况下进行修改。我们的App Wrapper工具包用于向商业不安全的应用程序添加安全功能。 Google Play中没有规定在没有Android源代码的情况下修改应用。此外,根据Google Android部署政策[10],在分发应用时只需注意几个方面,并且对应用修订和再分发限制没有任何限制。
		
	\section{总结}
		Android提供了基于权限的安全系统,但存在多个漏洞。为了解决这些漏洞,已经开展了通过静态策略进行权限控制和修补安全功能的研究。在权限控制的情况下,不可能为每个方法添加安全功能,并且每当发生权限控制时就生成许可警告窗口,从而导致应用的处理流程的开销和用户便利性的降低。在通过静态策略添加安全功能的情况下,每次更改安全策略时都应该重新打包应用程序。

		在本文中,我们提出了AppWrapper工具包来解决现有研究的局限性。建议的框架可以在需要字节码级别安全性的位置为不安全的应用程序添加安全功能,并管理安全功能,而无需通过动态策略管理进行应用程序重新打包。有一个优点是重新包装的阶段不是必需的。它还提供了一个日志视图管理界面,可根据应用程序进度流添加安全功能。在界面中,只需选择添加安全功能的位置,即可切换到策略管理界面并设置策略,从而提高用户便利性。

		Google Play Market的商业应用上的实验最大限度地减少了处理时间和文件大小的开销。尽管安全功能已添加到AndroidManifest.xml文件中所有活动类的方法单元中,但安全修补程序至少需要1到2.6秒,而.apk文件大小仅增加2.05%。未来的工作将包括修补易受攻击代码的能力。