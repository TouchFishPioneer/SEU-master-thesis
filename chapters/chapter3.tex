\chapter{撰写正文}

\section{研究生学位论文的一般格式与顺序}

根据《东南大学研究生学位论文格式规定》\cite{seugs2015rule}第一条之要求,研究生学位论文一般应由如下部分组成:

\begin{enumerate}
  \item 中文封面
  \item 中文页面
  \item 英文封面
  \item 论文独创性声明和使用授权声明
  \item 中文内容提要及关键词
  \item 英文内容提要及关键词
  \item 目录
  {\color{cyan} \item 符号、变量、缩略词等本论文专用术语注释表}
  \item 正文
  \item 致谢
  \item 参考文献
  {\color{cyan} \item 附录
  \item 中英文索引
  \item 作者简介(包括在学期间发表的论文和取得的学术成果清单)
  \item 后记}
\end{enumerate}
上述各部分得按照此顺序排列,其中{\color{cyan} 青色}标注的部分为可选部分。我们已经在第\ref{chp:initialization}章中介绍了上述列表中第1-3项关于封面中各条目的填写与生成方法。在本章中,我们将介绍如何撰写论文的正文以及其他部分。

\section{独创性与授权声明}

紧接在中英文封面后的应该是论文的独创性声明和使用授权声明,具体的文本内容请参考《学位论文独创性和使用授权声明》\cite{seugs2018license}。

本模板已经包含了对独创性声明和授权声明的自动生成,当编译引擎执行到

\begin{tcolorbox}
\begin{lstlisting}[language=TeX]
\makebigcover
\makecover
\end{lstlisting}
\end{tcolorbox}

\noindent 时会自动到目录下的seumasterthesis.cfg文件中寻找独创性与授权声明的预定义文本。

\section{中英文摘要}

《东南大学研究生学位论文格式规定》\cite{seugs2015rule}的第一条第二款中对论文摘要有如下要求:

~

{\color{black!45}
\noindent 论文摘要中文约500字左右,英文约200-300词左右,二者应基本对应。它是论文内容的高度概括,应说明研究目的、研究方法、成果和结论,要突出本论文的创造性成果或新的见解,用语简洁、准确。论文摘要后还应注明本文的关键词3-5个。关键词应为公知公用的词和学术术语,不可采用自造字词和略写、符号等,词组不宜过长。

\noindent 英文摘要采用第三人称单数语气介绍该学位论文内容,目的是便于其他文摘摘录,因此在写作英文文摘时不宜用第一人称的语气陈述。叙述的基本时态为一般现在时,确实需要强调过去的事情或者已经完成的行为才使用过去时、完成时等其他时态。可以采用被动语态,但要避免出现用“This paper”作为主语代替作者完成某些研究行为。}

~

打开工程目录下chapters文件夹中的abstract.tex文件,你就可以开始撰写论文的摘要。对于中文摘要,你会看到形如:

\begin{tcolorbox}
\begin{lstlisting}[language=TeX]
\begin{abstract}{生物学, 钓鱼, 铁憨憨}
我今天没吃饱。下面我将用70页的篇幅说明我今天为啥没吃饱,但是你看完后不一定能看懂。
\end{abstract}
\end{lstlisting}
\end{tcolorbox}

\noindent 这样的结构。在{\codefont $\backslash$begin\{abstract\}}之后的大括号里,你可以填写你的中文关键词。接下来直到{\codefont $\backslash$end\{abstract\}}之前的所有内容都将在编译时被视作你中文摘要的正文内容。英文摘要也与此类似,在abstract.tex文件中,你会看到形如:

\begin{tcolorbox}
\begin{lstlisting}[language=TeX]
\begin{englishabstract}{Biology, Phishing, Fucking Idiot}
I am not full today. I will use 70 pages to explain why I ain't full, but you may not understand after reading this piece of shit.
\end{englishabstract}
\end{lstlisting}
\end{tcolorbox}

\noindent 这样的结构,你可以把你的英文关键词和摘要填写在相应的位置。main.tex主文件通过:

\begin{tcolorbox}
\begin{lstlisting}[language=TeX]
%% ----------------------------------------------------------------------------
%%                              Chinese Abstract
%% ----------------------------------------------------------------------------
\begin{abstract}{\TeX, \LaTeX, 学位论文}
本文提出了一个新的东南大学 \LaTeX 硕士研究生毕业论文模板,并说明了如何更优雅地写出一篇漂亮而无用的文章。
\end{abstract}

%% ----------------------------------------------------------------------------
%%                              English Abstract
%% ----------------------------------------------------------------------------
\begin{englishabstract}{\TeX, \LaTeX, Thesis}
This article proposes a new Southeast University master degree thesis \LaTeX ~template and explains how to elegantly write an article which is beautiful but full of shit.
\end{englishabstract}

\end{lstlisting}
\end{tcolorbox}

\noindent 将abstract.tex文件作为外部依赖引入到主文件中,编译引擎在执行到该语句时会自动到chapters目录下寻找相应文本。

\section{论文章节及图表目录}
\label{sec:content}

本模板支持对所有章节和图表自动生成目录,在main.tex中:

\begin{tcolorbox}
\begin{lstlisting}[language=TeX]
\tableofcontents
\listofothers
\end{lstlisting}
\end{tcolorbox}

\noindent 语句控制了所有目录的自动生成,你不需要进行任何多余的操作。

\section{正文}
\label{sec:main_body}

我们在chapters目录下为你准备了若干名为chapterx.tex的文件,我们建议你将正文分章节书写在这些文件中。如果我们为你准备的6个章节文件尚且不能够满足你的章节数量需求,你可以继续在该目录下创建新的章节文件,并将其作为外部依赖添加到main.tex主文件中,就像这样:

\begin{tcolorbox}
\begin{lstlisting}[language=TeX]
...
\chapter{版权信息与更新记录}
\label{chp:version_license}

\section{版权信息}

本模板基于许元同学于2007年发布的 SEUThesis 和樊智猛同学于2016年发布的 SEUThesix,并在上述工作的基础上增加了一些新特性,并专注于对硕士研究生学位论文的支持。目前该模板能够同时支持学术型硕士研究生和专业型硕士研究生的学位论文。

~

\begin{tabular}{lll}
版权所有\copyright 2007--2012    & 许元      &(\url{xuyuan.cn@gmail.com})\\
                                & 宋翊涵    &(\url{syhannnn@gmail.com})\\
                                & 黄小雨    &(\url{nobel1984@gmail.com})\\
版权所有\copyright 2016          & 樊智猛    &(\url{zhimengfan1990@163.com})\\
版权所有\copyright 2019--2020    & 宋睿      &(\url{wurahara@163.com})\\
                                & 祁欣妤    &(\url{510371665@qq.com})\\
                                & 金星妤    &(\url{136204652@qq.com})\\
\end{tabular}

~

该程序是自由软件,你可以遵照自由软件基金会发布的《GNU 通用公共许可证条款第三版》来修改和重新发布这一程序,或者 根据您的选择使用任何更新的版本。我们希望发布的这款程序有用,但我们不对其可用性做任何程度的担保,甚至不保证它有经济价值和适合特定用途。更详细的情况请参阅\href{http://www.gnu.org/licenses/gpl.html}{《GNU 通用公共许可证》}。

我们基于GPL-v3发布该程序并不代表我们青睐于GPL许可证,相反我们认为GPL许可证是对开源社区的一种威胁和障碍。它如病毒般的传播条款将会极大限制基于GPL协议开发的自由程序的分发与使用。我们使用GPL许可证仅仅是因为我们所基于的程序使用了它,而GPL-v3规定所有对使用了该许可证的程序的二次分发和代码利用都必须使用同样的许可证开放源代码。但本模板的所有开发者和维护者都一致认为有必要声明我们对GPL的厌恶和反对。

\section{更新历史}

\begin{description}
  \setlength{\itemsep}{2pt}
  \setlength{\parsep}{2pt}
  \setlength{\parskip}{2pt}
  \item[3.4.3] 添加了 Makefile 文件和 GNU Make 支持,并在手册中添加了相关说明。
  \item[3.4.1] 修正了专业型硕士研究生学位论文的相关设定和渲染格式,并在手册中添加了相关说明。
  \item[3.3.5] 调整了文献引用的格式;调整了BST文件的若干细节,使之符合东南大学研究生院参考文献引用标准;调整了取消链接着色后的边框显示。
  \item[3.3.3] 将一些专有名词从CLS文件中抽出并放置于CFG文件中,调整了CFG文件的结构;修正了论文A3封面书脊中西文混排时西文基线高度偏低的问题,并在手册中添加了相关介绍。
  \item[3.3.1] 添加了对专业型硕士研究生学位论文的支持;调整了表格框线的线型和边距。
  \item[3.2.5] 添加了对模板参数的介绍;添加了对子图的支持。
  \item[3.1.1] 删去了CLS文件的一些暴露参数;添加了针对Windows操作系统的编译脚本;撰写文档声明,并正式开放源代码。
  \item[3.0.3] 调整了参考文献渲染格式,使其符合GB/T 7714-2015国家标准。
  \item[3.0.1] 大幅调整了CLS文件的结构与布局,取消了对博士研究生学位论文的支持。
\end{description}

\input{chapters/chapter7}
\input{chapters/chapter8}
...
\end{lstlisting}
\end{tcolorbox}

本模板对文章的章节结构支持到了小节级别。如果你想创建新的章,请使用:

\begin{tcolorbox}
\begin{lstlisting}[language=TeX]
\chapter{母猪的产后护理}
\end{lstlisting}
\end{tcolorbox}

\noindent 这样的命令,它将为你新建一个名为“母猪的产后护理”的章。节与小节的创建方法与此类似:

\begin{tcolorbox}
\begin{lstlisting}[language=TeX]
\section{母猪产后抑郁了怎么办}
\subsection{母猪的心理疏导}
\end{lstlisting}
\end{tcolorbox}

\LaTeX 相比于Microsoft Word等文本编辑器的优势在于,它对交叉引用和自动编号的支持极其自然和友好,以至于你完全不需要耗费精力管理相关的内容。比如说你在正文中需要引用前文的某个章节,你只需要在该章节处添加一个标签,就像这样:

\begin{tcolorbox}
\begin{lstlisting}[language=TeX]
\chapter{母猪的产后护理}
\label{chp:postnatal_care}
\end{lstlisting}
\end{tcolorbox}

\noindent 随后如果你想要在其他部分引述该章节的内容。你只需要在相应位置插入该章节的标签,就像这样:

\begin{tcolorbox}
\begin{lstlisting}[language=TeX]
在第\ref{chp:postnatal_care}章,我们介绍了如何对母猪进行产后护理。那么萨达姆是如何根据该经验做好对美国的战斗准备的呢?
\end{lstlisting}
\end{tcolorbox}

\noindent 那么在论文编译时,上面的引用就会被自动替换为相应章节的名称,就像这样:

\begin{tcolorbox}
\begin{lstlisting}[language=TeX]
在第三章,我们介绍了如何对母猪进行产后护理。那么萨达姆是如何根据该经验做好对美国的战斗准备的呢?
\end{lstlisting}
\end{tcolorbox}

\noindent 在论文的撰写过程中请活用该功能,它能为你提供许多方便。

\section{致谢}

你可以在chapters目录下的acknowledgement.tex文件中写下你对任何人的任何感谢,这是学位论文中你唯一可以恣情释放的地方,请尽情享受吧。

\section{参考文献}

和目录与引用类似,本模板支持对参考文献列表的自动生成,在main.tex中:

\begin{tcolorbox}
\begin{lstlisting}[language=TeX]
\thesisbib{seumasterthesis}
\end{lstlisting}
\end{tcolorbox}

\noindent 命令实现了这一功能。关于如何引入参考文献以及如何在正文中引用特定的参考文献条目,我们还将在第\ref{chp:bib}章进行详细地介绍。

\section{附录}

根据《东南大学研究生学位论文格式规定》\cite{seugs2015rule}的第一条第八款,你可以将正文有关的原始数据明细表、较多的图表、程序源代码、过长的公式推导等不宜置于正文部分的文本放在附录中。你可以在chapters目录下的appendix.tex文件中添加你的附录。如果你有多个附录的话,可以通过在该文件中新增:

\begin{tcolorbox}
\begin{lstlisting}[language=TeX]
\chapter{沙漠风暴行动D日攻击计划表}
\end{lstlisting}
\end{tcolorbox}

\noindent 来添加附录项。每个附录项都将被以大写英文字母编号和排序,并均会新起一页。除此之外,附录内容的撰写方法和正文基本一致。

如果你的论文不需要安排附录,请在main.tex主文件中删去或注释该行:

\begin{tcolorbox}
\begin{lstlisting}[language=TeX]
\appendix
\newtheorem{theorem}{定理}

\chapter{欧几里得第二定理的证明}
\label{appendix:apps}

	\begin{theorem}
		欧几里得第二定理(素数有无穷多个)\\
		证明:用反证法。假设素数有有限个($N$个),记为$p_1,p_2,\dots,p_N$。则我们构造一个新的数,
		
		\[n=p_1p_2\dots p_N+1.\]
		
		由于$p_i,i=1,2,\dots,N$为素数,则一定不为$1$。于是对于任意的$p_i,i=1,2,\dots, N$,有
		
		\[p_i\not|n\]
		
		这表明,要么$n$本身为素数,要么$n$为合数,但是存在$p_1,p_2,\dots,p_N$之外的其他素数能够将$n$进行素因子分解。不管哪种情况,都表明存在更多的素数。定理得证。\qed
	\end{theorem}

\chapter{$\sqrt{2}$是无理数的证明}
	\begin{theorem}
		$\sqrt{2}$是无理数。\\
		证明:用反证法。假设$\sqrt{2}$是有理数,则可表示为两个整数的商,即$\exists p,q, q\ne0$
		
		\[\sqrt{2}=\frac{p}{q}\]
		
		不失一般性,我们假设$p,q$是既约的,即$\gcd(p,q)=1$。对上式两边平方可得\\
		
		\begin{align*}
			2& =\frac{p^2}{q^2}\\
			p^2&=2q^2.
		\end{align*}
		
		表明$p^2$为偶数,因此$p$为偶数,记$p=2m$。则
		
		\begin{align*}
			p^2&=4m^2=2q^2\\
			q^2&=2m^2.
		\end{align*}
		
		表明$q$也为偶数,因此它们有公共因子$2$。这与它们既约的假设矛盾。定理得证。\qed
	\end{theorem} 
\end{lstlisting}
\end{tcolorbox}

\section{作者简介}

你可以在作者简介部分简要介绍你的姓名、出生年月、籍贯等基本信息,并简要列举你在攻读学位阶段参与的科研课题、发表的学术论文、获取的发明专利或著作权,以及其他的一些科研成果。《东南大学研究生学位论文格式规定》\cite{seugs2015rule}的第一条第十款建议硕士研究生将该部分限制在1000字以内,博士研究生则在2000字以内。

我们在chapters目录下的resume.tex文件中为你准备了一份模板,你可以根据你的实际情况进行修改。